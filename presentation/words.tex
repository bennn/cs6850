% Created 2013-12-01 Sun 21:53
\documentclass[11pt]{article}
\usepackage[utf8]{inputenc}
\usepackage[T1]{fontenc}
\usepackage{fixltx2e}
\usepackage{fullpage}
\usepackage{graphicx}
\usepackage{longtable}
\usepackage{float}
\usepackage{wrapfig}
\usepackage{soul}
\usepackage{textcomp}
\usepackage{marvosym}
\usepackage{wasysym}
\usepackage{latexsym}
\usepackage{amssymb}
\usepackage{hyperref}
\tolerance=1000
\providecommand{\alert}[1]{\textbf{#1}}

\title{words}
\author{λ}
\date{\today}
\hypersetup{
  pdfkeywords={},
  pdfsubject={},
  pdfcreator={Emacs Org-mode version 7.8.11}}

\begin{document}

%% \maketitle

%% \setcounter{tocdepth}{3}
%% \tableofcontents
%% \vspace*{1cm}
\section{Title}
\label{sec-1}
\begin{description}
  \item[Ben:] Hi I'm Ben this is Hersh and we've been studying Herd Behavior in Financial Markets
\end{description}
\section{Overview}
\label{sec-2}
\begin{description}
  \item[Ben:] So first, what is herding? Herding is when investors ignore their private information and instead trade based on what the crowd does.
       That is, instead of using some private signal about what a stock is worth, they instead base their actions off what market competitors are doing.
  \item[Ben:] There are a number of suggested reasons for why this happens, for example to minimize risk or to supplement poor or missing private information.
       One instance might be a trader whose compensation is tied to his performance on the floor. Rather than taking risks, he buys safely, with the crowd, where the likelihood of big losses is minimized.
  \item[Hersh:] Herding is relevant in a variety of ways. Recent stock bubbles and crashes were the result of herd behavior, and smaller example of herding occur on a daily basis.
         (add another sentence?)
  \item[Ben:] There exists a fair amount of work studying herding as a phenomenon in traditional, human-driven markets.
       On one hand there are the papers which give evidence that herding exists in some targeted market and attempt to explain why, and other works give a mathematical model explaining and predicting herd behavior.
       These models have focused on, for example, the heavy tails characteristic of herding and the importance of private signals or proportion of informed traders.
       However they do not really apply modern HFT markets, which leads to our contribution
\end{description}
\section{Our contribution}
\label{sec-3}
\begin{description}
  \item[Hersh:] We want to study herding in the context of high-frequency markets, where the trading is done primarily by algorithms.
         These are not subject to the same incentives or emotions as human traders, yet still exhibit herd behavior.
         You can imagine why--it makes sense to leverage others' actions as a source of public information--but there's a lack of formalism about these intuitions.
  \item[Hersh:] Additionally, we think that previous herding models have fallen short by not using trade volume as a major predictor for herding.
         If a buyer purchases a larger-than-normal quantity of one stock, we expect other traders to be more influenced by this action--more likely to herd in that direction.
         So we are developing a new model to account for this.
  \item[Ben:] For data, we have been using the Interactive Broker API. 
       Over the past few weeks we've been scraping their data, which gives us prices and quantities updated every 5ms or so.
       Graphing the best bid and ask along with the price-weighted average of all trades, we expect to see the market average shift in the direction of high-volume trades.
\end{description}
\end{document}
