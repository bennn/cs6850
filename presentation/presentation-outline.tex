% Created 2013-11-27 Wed 21:02
\documentclass[11pt]{article}
\usepackage[utf8]{inputenc}
\usepackage[T1]{fontenc}
\usepackage{fixltx2e}
\usepackage{graphicx}
\usepackage{longtable}
\usepackage{float}
\usepackage{wrapfig}
\usepackage{soul}
\usepackage{textcomp}
\usepackage{marvosym}
\usepackage{wasysym}
\usepackage{latexsym}
\usepackage{amssymb}
\usepackage{hyperref}
\tolerance=1000
\providecommand{\alert}[1]{\textbf{#1}}

%% \title{presentation}
%% \author{λ}
%% \date{\today}
%% \hypersetup{
%%   pdfkeywords={},
%%   pdfsubject={},
%%   pdfcreator={Emacs Org-mode version 7.8.11}}

\begin{document}

%% \maketitle

\setcounter{tocdepth}{3}
%% \tableofcontents
\vspace*{1cm}
\section*{Slides}
\label{sec-1}

\begin{enumerate}
\item Title page. Ben \& Hersh, ``Herd Behavior in High-Frequency Markets''
\item Quick definition of herding. ``When traders ignore their private information and follow competitors' actions''
\item Relevance. Herding is a strategy used to minimize risk and also as a more reckless gamble. i.e. bubbles.
\item Citations. Empirical evidence showing herding exists in real markets, theoretical work modeling this behavior.
\item Shortcomings of prior work. Not really applicable to HFT data, focuses on human elements (so doesn't explain algorithms), volume ignored
\item Our contribution. Been collecting price/volume data, going to experiment with a model which assumes trade volume as a key predictor.
\item Almgren graph showing what we expect to see?
\item The end. We don't have any results yet so this is the last slide. Maybe we should get results.
\end{enumerate}

\end{document}
