\documentclass{article}
\usepackage{cs6850}

\renewcommand\maketitle{
\begin{flushleft}
{%
\large{%
Benjamin Greenman,
Hersh Mehta
\hfill
CS6850: Reaction Paper
\\\texttt{\{blg59, hpm36\}} \hfill \today
}}\\
\hrulefill
\end{flushleft}
}

\begin{document}
\maketitle

\section{Introduction}
%% 2013-10-24 TODO needs review
%% Introduce topic
%% - herding y aherd
Financial markets are commonly modeled as a set of independent actors who combine public and private information in order to make investments.
A minimum of information sharing is expected, perhaps by analyzing an individual's past spending, but for the most part private information remains so.
At no point does all private information become public, creating an even playing field where rational agents will make identical decisions.
The market is always characterized by information asymmetries.

Naturally, these models carry the assumption that individuals' choices are strongly guided by their private information.
The publicly known data is what prevents them from making poor decisions, keeping them updated with how market action has changed the dynamics of the system.
On the other hand private information gives investors an edge, helping them make a decision that is likely to succeed.
Wherever it applies, one expects players to make choices based upon their private data, rather than the publicly known information.
At any rate, this is what a large number of investment models are based on\textemdash an assumption that rational participants are guided above all else by their private information.
%% Whenever an investor is observed to buy a stock that appears weak given the public information, one infers that their private information provided additional details.

However, this assumption has been contradicted by empirical studies. 
%% TODO cite sources
Numerous studies, have confirmed that investors do not always act as individuals, guided by private information, but rather exhibit a tendency for collective action\cite[174]{cont-bouchad}.
Investors, in practice, herd.
They are influenced by the actions of other participants to the point of abandoning their private knowledge and converging on a single decision.
%% Herd mentality takes control; individuals follow the crowd rather than their own information.

Reasons for herding are myraid, and will not be discussed here.
Additionally, we elide discussion of the outcomes of herding and of the circumstances in which herding becomes the optimal strategy.
Rather, we concern ourselves with modeling herd behavior in investment markets.
In particular, we analyze the manner in which previous studies have characterized herding, note where they have succeed, and identify where they may be improved.

\section{Background}
%% 
Our approach in choosing papers to read for this topic was to find one established, canonical paper that has had a large influence on future works, and a newer paper offering strategies relevant today.
From the former, we sought a decisive summary of the phenomena and problem, and an intuitive uncluttered model upon which to base theoretical results.
The latter paper we hoped would summarize research up until the present, identify successes and failures, and show us the current state of the research.
To this end we have read Cont and Bouchaud (2000), which is a well-cited work which presents the first model of herding based on a randomized graph, and Lin et. al. (2009), which provides a good critique of existing models and offers a suggestion for a new model addressing the demands of high-frequency trading networks.
For additional background we have read Bikhchandani and Sharma (2001), which we reference where applicable.
The two papers provide a nice constrast with one another, showing how the literature has evolved over time.

%% Summarize sources
%% - what's the main technical content
%% - why's content interesting in relation to the corresponding section of the course
%% - what are some weaknesses of the papers, how can they be improved
%% - what are some promising future questions, how could they be answered (what're we gonna do for research)
\section{Review of Cont \& Bouchaud}
hay 

\subsubsection{Summary}
\subsubsection{Relation to coursework}
\subsubsection{Weaknesses}
\subsubsection{Questions}

\subsection{Review of Lin et. al.}
\subsubsection{Summary}
\subsubsection{Relation to coursework}
\subsubsection{Weaknesses}
\subsubsection{Questions}

\section{Proposal}
%% Give a proposal for a model, or a new algorithm. Can extend / vary something from papers

High-frequency trading is a relatively new phenomena, but has revolutionized financial markets. 
Since its emergence in XXXX, 

We are interested in seeing how trading algorithms are affected by herding, and in providing a suitable model for this economy.

\section{Analysis}
%% Test of a model or algorithm. Our own or something from papers, done on dataset or simulated data
\subsection{Goals}
\subsection{Dataset}
\subsection{Measure}
%% A good measure should depend on the data.
%% Maybe not 'be developed with this specific input in mind', 
%% but at least have variables contingent on the data
\subsection{Results}

\section{Discussion}
%% 2013-10-24: I am happy. Maybe cut this whole thing later but for now I am happy.
We have thus far considered herding the irrational, or at least unexpected behavior, assuming that investors follow their private information as the norm and degenerate into the herd mentality in exenuating circumstances.
Indeed, this is the the viewpoint contemporary models have embraced: herding is measured as deviation from some expected investment strategy, identified by similarity across different parties. 
One wonders whether this is a valid assumption. 
Making decisions based off the actions of fellow investors is not so far removed from making decisions based on publicly available information.
Historical trends, which are in effect what herd participants follow when making their decisions, are simply another form of public information. 
The herd mentality is more or less what we would expect to see in a completely fair market, without any information asymmetries. 
We might then refine the initial question to assume that herding will occur and instead ask how significant a role market trends play in investors' decision-making\textemdash if the herd action exerts a disproportionately large influence on buyer and seller behavior\textemdash but it seems strange, on review, to assume that players are not greatly influenced by their competitors actions.

%% In this spirit
Perhaps a better phemonena to explore are the circumstances in which an investor ignores \emph{public} information and instead follows private knowledge or intuition.
A contrary model, which instead of tracking similarity measures differences between investors, might provide more insight to the workings of these markets.  
Assuming that investors follow the public information and each others' decisions exclusively might permit us to draw conclusions about the influence or extent of participants' private information.
This is a naive investment strategy, subject to gaming and easily misled, but may in fact be a better generalization for how investors function in practice.


\section{Conclusion}
%% Summarize papers again
%% Summarize our thoughts, our contribution
%% Lay out future work

\bibliographystyle{amsplain}
\bibliography{reaction-paper}

\end{document}
