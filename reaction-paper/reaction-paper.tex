\documentclass{article}
\usepackage{cs6850}

\renewcommand\maketitle{
\begin{flushleft}
{%
\large{%
Benjamin Greenman,
Hersh Mehta
\hfill
CS6850: Reaction Paper
\texttt{\{blg59, hpm36\}} \hfill \today
}}\\
\hrulefill
\end{flushleft}
}

\begin{document}
\maketitle

\section{Introduction}
%% Introduce topic
%% Name sources
%% Say what we want to do

\section{Review}
%% Summarize sources
%% - what's the main technical content
%% - why's content interesting in relation to the corresponding section of the course
%% - what are some weaknesses of the papers, how can they be improved
%% - what are some promising future questions, how could they be answered (what're we gonna do for research)
\subsection{Source A}
%% TODO choose source...
\subsubsection{Summary}
\subsubsection{Relation to coursework}
\subsubsection{Weaknesses}
\subsubsection{Questions}

\subsection{Source B}
%% TODO choose source...
\subsubsection{Summary}
\subsubsection{Relation to coursework}
\subsubsection{Weaknesses}
\subsubsection{Questions}

\section{Proposal}
%% Give a proposal for a model, or a new algorithm. Can extend / vary something from papers

\section{Analysis}
%% Test of a model or algorithm. Our own or something from papers, done on dataset or simulated data

\section{Conclusion}
%% Summarize papers again
%% Summarize our thoughts, our contribution
%% Lay out future work

\bibliographystyle{amsplain}
\bibliography{reaction-paper}

\end{document}
