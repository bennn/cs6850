\section{Model}
\label{model}
%% Our herding model, explanation of components

%% We like their use of three values. It makes sense because signals have to be U-shaped to herd
%% We almost like their herding criteria
%% - keeping the 'do what you didn't do at original time', because it makes sense that private signals don't change so much over the day
%%   we are working with short timesteps here, so we will keep the private signals
%% - so then first condition is `did X initially'. Second condition is `did Y'
%% - Third condition is that market herds. We change our conception of this. 
%%   (I don't quite agree here. I think the old one--most transactions are X--is fine)
%%   We have a super-additive function that determines when the price-weighted buys/sells tend towards buy or sell

%% Goal is for our model to predict the herding clearly visible in the simulated or experimental graphs.
%% - not certain how this works with simulated graphs

Here we describe the parameters we have added to and altered within the existing model.

\subsection{Types of Traders}
\todo{will have informed and ??? traders. Want to replace noise traders.}

%% TODO we need a function
\subsection{Volume-Rated Herding Measure}


\subsection{Herding Measure}
These defined, we present our definition of herding. 
A buyer $B$ is said to herd in time $t$ if:
\begin{itemize}
\item He eats nachos.
\item Wears sunglasses and a pink shirt.\footnote{Unless it's a Saturday, in which case the shirt must be cyan.}
\item Says `bro' 3 or more times in one sentence ever.
\end{itemize}
