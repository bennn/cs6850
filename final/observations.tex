\section{Herding in a High-Frequency Setting}
\label{observations}
%% What is HFT
High-frequency trading is in-formally defined as the strategy of placing ``a large number of orders'' that can be ``rapidly canceled'' and are ``held for a short while''~\cite[2]{ahlstedt2012high}.
Here ``a short while'' can range from a few milliseconds up-to half an hour.
These trades happen so quickly and so frequently that in the U.S. equity markets the average holding period for a stock is 22 seconds~\cite[2]{ahlstedt2012high}.
This means trades can occur in much more rapid succession, giving a succession of small gains the potential to accumulate into large profits.

%% Our hypotheses about HFT.
For the purposes of this endeavor, we assume that herding does occur in HFT settings. 
This assumption is reasonable given the observations of prior works; it has been shown that herding occurs in emerging high-frequency markets and the rational motivations for herding indicate that optimizing algorithms will identify and take advantage of situations where herding is likely to lead to the best outcomes~\cite{lin,boortz,bikhchandani}.
However, it is worth mentioning that this assumption is not a given.

Financial data, particularly HFT data, is valuable and difficult to come by.
Free sources, aside from what governments demand, are practically non-existant as proprietary services run a lucrative business tracking and distributing information on exchanges.
Yet even these suffer insufficiencies.
For one, some markets evade collection.
European markets are sufficiently transparent but U.S. markets are privatized, with a significant portion of trades occurring in dark-pools, and thus more difficult to get information for~\cite{beny2005insider}.
Foreign exchange markets present a different challenge in that there is no central governing body documenting all exchanges~\cite{sarno2001official}.
Within the Forex markets, trading occurs amongst various liquidity providers, and there aren't individual exchanges, such as the NASDAQ or Dow Jones for stocks, where these transactions are cleared.
So our research is limited by the sources available.

For data which is collected, one faces the additional problem of completeness. 
There are varying definitions of ``high-frequency''; instead of milli- or micro- second data, some purportedly HFT sources offer data on the order of tenths of seconds or even whole seconds.
Another issue is that quantities are not always reported alongside prices, and the buyer and seller are frequently anonymized.
Without quantities, the interesting predictions we can do are sorely limited.
Anonymization might not be such an issue, but traders in many cases are not even given unique identifiers.
The data may not give a means of attributing the purchase of a few different securities to the single buyer who purchased them all.

Fortunately, one member of our group had an account with Interactive Brokers and we used this account to gather our data.
Interactive Brokers (IB) is a service providing a web interface into securities markets~\cite{ib}.
Founded in 1977, the site ``has grown internally to become one of the premier securities firms with over \$4.8 billion in equity capital.''
For our purposes, Interactive Brokers offered price and quantity data approximately every five milliseconds.
This was frequent enough for our purposes, and furthermore the data associated quantity information with every price.
Hence IB provided us a record of recent transactions and a bit of information upon which to base conjectures.

That said, this data was certainly not complete.
Though we were excited to have frequent updates and quantities, IB only provided and updated an order book: a summary of the current bids and asks.
An order book consists of rows of the current bid and ask prices.
These rows are ordered by price. 
The highest bid and the lowest ask are at the top of their respective columns. 
Associated with these prices are the quantities participants are seeking to work in.
These quantities are not associated with individual traders but rather with the price as a whole.
This make accounting for transactions difficult, as we cannot tell who participated in a given trade.
An additional complication is the way in which data is reported. 
Changes to the order book appear one at a time, in sequence.
That is, if trader A buys $q$ units of trader B's stock at price $p$, the order book is first changed to show that the aggregate buyers are seeking to buy $q$ less stock at price $p$ and second to show that the sellers have $q$ units less to offer at price $p$.
The transaction does not occur in one continguous step, thus making it impossible to discern market entry/exit from actual trades.
This is a source of concern because in HFT, over 90\% of market actions are entries and exits~\cite{almgren}. 
Unfortunately, this is a major source of inaccuracy in our analysis; we revisit this problem later.
Furthermore, we have no means of associating one trade with a certain market participant. 
Changes in price and quantity appear without an identifier, thus, we cannot apply the models of, for example, Park \& Sabourian or Boortz et~al., which monitor individual traders over a period of time~\cite{boortz,park2011herding}.

Regardless, Interactive Brokers provided a sufficient data source for our purposes.
Specifically, we looked at the spot EUR/USD exchange rate.
After obtaining full access to the API and writing the initial plumbing code to interface with their quote dissemination servers, we began collecting data in early November.
Although these early collections were admittedly sparse, they were useful for getting a general feel for our target market, which was foreign exchange currencies between the United States, Asia, and Europe.

Beginning around Thanksgiving break, we switched gears and began extracting data daily and for extended periods, building weekly sets of data.
We base our experiments and presentations in Section \ref{experiment} off one such week's worth of data, collected starting December 3.

\subsection{Conjecture}
High-frequency data differs from traditional data in that it moves much faster and it generated by precise machines, but the two still share the same underlying structure.
At the end of the day, the objective is to make money given limited information about the market.
Each agent has access to public information as well as their own private signal; a key objective of agent A is to learn as much as possible about the signals of the other agents so to make the most informed decision possible.
Thus if we begin with what we know about traditional trading settings and add alterations to handle the peculiarities of HFT we can determine a predictor for price movements and, ultimately, herding.

Considering traditional markets, we recall one reason participants would herd was to supplement insufficient or untrusted private information~\cite[291]{bikhchandani}.
In other words, the participant relies more on outside data when less confident about the private signal.
Conversely, participants who are very confident about their private signal are unlikely to herd. 
In fact, given sufficient confidence and opportunity for profit they may make larger-than-normal trades. 
Hence we take large-quantity trades to be an indicator of herding. 
Applying this intuition to HFT markets, we note that the average quantity of high-frequency trades should be low.
Trading algorithms make money by providing liquidity to other participants~\cite{guo2012high}.
They make money through many small, low quantity trades; therefore, we expect to see a low average trade quantity in an HFT market. 
This characteristic in mind, a large quantity trade would be quite out of the ordinary\textemdash something to take note of.
Here we have the basis for our first conjecture regarding high-frequency data:

\vspace{0.5cm}
\hspace{1cm}  \emph{High-quantity trades significantly influence HFT price trends}
\vspace{0.5cm}

That is, we expect trading algorithms to take note of and modify their position in accordance with the high-volume, presumably-high-confidence trades.
For an algorithm bids heavily on a particular future is an uncommon event, worth paying attention to.
We hypothesize that traders in the HFT market will follow this decision pattern. 
If they are at all inclined to herd, they will do so following the direction of large-quantity trades.
