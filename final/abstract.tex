\label{abstract}

Herd behavior in traditional commodities markets is a fairly well-studied subject.
The earliest relevant literature dates back to the late 1980's \cite{bikhchandani}, and since then a number of experiments and theoretical models have observed and characterized the behavior.
These results established that herding does indeed exist and offer reasonable suggestions motivating and modeling the phenomena.
However, high-frequency trading (HFT) has overhauled markets, and little to no published literature explores herding in this context.

We attempt a formal analysis of herding in a HFT setting. 
Using order book data we attempt to predict price shifts and herd behavior and then discuss our results, their implications, and our hypotheses.
