\section{Background}
\label{background}

%% Formal definition of herding
%% Evidence papers, who's seen herding happen
Herding, as previously defined, is when traders ignore their private information and instead buy as a crowd.
This is a fairly well-known and well-documented phenomena, particularly in traditional (pre-HFT) financial markets, and we recount some of the relevant literature here.

For the most part, works on herding can be put into one of two categories. 
Either they give an example of herding, identifying and giving reason for an observed behavior, or they present a theoretical model seeking to characterize herd behavior.
We use this apparent split to guide our discussion, considering first the works that establish herding as a real phenomena and second the more academic hypotheses and conjectures that have stemmed from thought on herding.
However, we make an exception for one notable model, the VPIN model of O'Hara, Easley, and Lop\`ez de Prado. 
This model is particularly influential for our work and how we interpret the data, so we give it special attention here. 

\subsection{Evidence of Herding}
%% TODO
Many papers in recent memory document the existence of herd behavior in traditional financial markets.
%% scharfstein1990herd
The earliest we know of is Shiller \& Pound (1987).
They conducted a survey of institutional investors, seeking to determine the influence of direct, personal communications on their decisions~\cite{shiller1989survey}.
Results strongly indicated that investors follow one anothers' lead, which ultimately led to the realization that they were acting as a herd.
Scharfstein \& Stein (1990) explore herd behavior among firm managers, providing both rationale and an explanatory model~\cite[3]{scharfstein1990herd}.
They also connect their model to stock markets, arguing that the financial sector as another location where their insights apply.
Grinblatt, Titman, \& Wermers studied mutual funds for evidence of herding and found that a large number of investors are momentum buyers/sellers; that is, they buy past winners and sell past losers~\cite{grinblatt1995momentum}.
This behavior is significant because it means the entire space of traders eventually converge on the same set of winners and losers.
Lakonishok, Shleifer, and Vishny (1991) continue this line of reasoning in their exploration of herd behavior among tax-exempt funds~\cite{lakonishok1992impact}.
Other examples include Trueman's study of herd behavior among analysts~\cite{trueman1994analyst}, Golec's observations on herding~\cite{golec1997herding}, and Sias's (2004) study of institutional herding~\cite{sias2004institutional}.
%% Devenow \& Welch's 
The most recent example we encountered is that of Lin, Tsai, and Sun (2009), which examines herd behavior within the emerging Taiwanese market~\cite{lin}.

In summary, there is a large body of evidence that herding is a widespread phenomena among human traders, and a small but growing pool of work noting herd behavior in HFT data.

\subsection{Theoretical Models}

%% \subsubsection{Cont \& Bouchaud}
%% %% Heavy-tailed distributions
%% \subsubsection{Park \& Sabourian}
%% %% TODO how they proved U-shape
%% \subsubsection{Lin, Tsai, Sun}
%% %% 2013-12-13: these are personal notes, not certain how to incorporate
%% %%             prettymuch that there are non-human reasons to herd..
%% \subsubsection{Boortz et~al.}
%% %% TODO herding model
On the theoretical side, we have a variety of models that seek to capture the essence of herd behavior.
An early, na\"ive model is that of Bikhchandi (1992), which gave the simplest effective model of herding~\cite{bikhchandani1992theory}.
They have investors enter the market one-by-one, in turn making decisions of whether to buy, sell, or hold based on their private signal, competitors' actions, and the transaction history.
Building upon this model, Orl\`{e}an (1995) provided an asynchronous simulated market~\cite{orlean1995}.
However, they gave all players an equal likelihood of copying one another, a misgiving which was corrected by Cont \& Bouchad (2000)~\cite{cont}.
%% Bak, Paczuski, \& Shubik (1997) give a detailed and complex model for stock markets with many agents, furthermore accounting for the non-Gaussian statistics that appear in the presence of herding, but this model was arguably too parameterized and complex to be readily applied to experiments~\cite{bak1997price}. 

Models dealing with HFT data specifically include that of Park \& Sabourian (2011) and Bootrz et~al. (2012)~\cite{park2011herding,boortz}.
These focus on the transactions surrounding a single stock over time.
Traders receive a private signal in the beginning of the simulation and the model observes whether they change from buying to selling (or vice-versa) at a later point in time.
This model is of special interest to us as we conduct our study of HFT data, and we recall the lessons of Boortz et~al. as we study our own dataset.

\subsection{VPIN}
The Volume Synchronized Probability of Informed Trading (VPIN) model, invented by O'Hara, Easley, and Lop\`ez de Prado, is a measure of the relative liquidity in a market~\cite{vpin}.
Designed to inform market makers in an HFT setting, the model gives a measure of how likely it is that informed trading is occurring within the market.
This helps the market maker identify when it is providing liquidity at a loss. 

VPIN builds off a previous model, the PIN, by bucketing actions by volume. 
This is appropriate for high-frequency data, where a standard timestep no longer provides an accurate characterization of the market.
Over the course of one second, a vast number of trades might take place.
Or, depending on the time of day, extremely few trades might occur\textemdash HFT makes the frequency unpredictable.
Choosing to group by volume ensures a stable unit of measure and moreover is not dependent on non-observable parameters and updates in stochastic time~\cite[4]{vpin}.

This model is essential to our analysis in Section \ref{experiment}, and so we recount it briefly here. 
In particular, we are concerned, in our experiment, with predicting price movements for the purpose of tracking herd behavior. 
Using the VPIN metric gives a solid theoretical foundation upon which we can base our own inferences and conjectures.

Trades occur in the market over a period of time intervals; at each interval, traders arrive according to some Poisson process and an information event occurs with probability $\alpha$.\footnote{The information may be good or bad and the traders may be informed or uniformed, but these details are not essential to our use of the VPIN.}
The liquidity provider is an important player.
At each time $t$ he holds certain beliefs $P(t) = (P_n(t), P_b(t), P_g(t))$ about whether the news affecting the informed traders was good, bad, or neutral.
He updates his beliefs at each time interval according to the expected value of the asset:
\begin{align*}
  E[S_i |t] = P_n(t)S_i^* + P_b(t)\underline{S}_j + P_g(t)\overline{S}_i
\end{align*}
$S_i^*$ is the prior expected value of the asset.

With this value, the liquidity provider can determine the expected Bid $B(t)$ and Ask $A(t)$.
These are defined in terms of $\mu$, the proportion of informed traders, and $\varepsilon$, the proportion of uniformed traders, as follows:
\[\begin{array}{r l}
  B(t) = E[S_i|t] - 
    \frac{\mu P_g(t)}{\varepsilon + \mu P_g(t)}
    (E[S_i]
    |t -
    \underline{S}_j) 
    & 
  A(t) = E[S_i|t] + \frac{\mu P_g(t)}{\varepsilon + \mu P_g(t)}(\underline{S}_j - E[S_i|t])
\end{array}\]
Read these as the expected value of the asset conditional on someone wanting to buy the assert from the liquidity provider~\cite[10]{vpin}.

Lastly, we have the probability $PIN$ that an order came from an informed trader. 
This value is key to decisions:
\begin{align*}
  PIN = \frac{\alpha\mu}{\alpha\mu + 2\varepsilon}
\end{align*}
Moving forward, we understand that the market maker is continuously wary of the trades it observes.
It constantly tries to determine whether an informed or noise trader made the order, and makes future decisions accordingly. 
If there seem to be many traders with significant private information, invisible to the maker, it will step out of the market.



\subsection{Comments}
In summary, we have seen that herding does indeed exist withing both purely human and transitionary human-and-HF settings.
Furthermore, we have discussed an array of models offering guiding insight for predicting herd behavior, each correcting an old belief or identifying a new, previously unconsidered parameter.
We turn now towards herding in an HFT setting, describing the challenges we expect to see and our plans for addressing them.
%% add a word on the vpin?
