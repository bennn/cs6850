\section{Introduction}
\label{introduction}
%% 2013-12-16: this is probably GOOD
%% Very similar to reaction paper or topic proposal, I hope.
Financial markets are commonly modeled by a set of independent actors who combine public and private information in order to make investments.
Public information is, as the name suggest, freely available to all decision-makers.
On the other hand, private information is unique to each.
A reasonable assumption when studying trade markets is that a minimum of information sharing occurs but for the most part private information remains so.
At no point does all private information become public, thus the market is characterized by inequality and information asymmetry, with players vying to learn their opponents' secrets.

These models carry the assumption that individuals' choices are strongly guided by their private information.
However, this assumption has been contradicted by empirical findings. 
Numerous studies have found that investors do not always act as individuals, guided by private information, but rather exhibit a tendency for collective action.
Participants occasionally disregard their own private signal and instead buy with the crowd.

%% What is herding
This behavior\textemdash ignoring private information and instead basing decisions off competitors' actions\textemdash is known as herding.
Investors herd when they buy as a crowd, and a market in which this occurs is said to exhibit herd behavior.
We are interested in identifying, classifying, and predicting herd behavior in financial markets.
%% Why do we care
This has a variety of applications, most notable of which is to predict the formation of bubbles and prevent their subsequent collapses. 
Recent financial crises like the Dot-Com bubble and the Subprime Mortgage crisis provide strong motivation for controlling herd behavior.
On a more micro scale, being able to predict when a market will herd has money-making potential for trading algorithms.
By judiciously providing liquidity to influence herd events, and algorithm can maintain a slight edge on the market.

%% Our goal
Herd behavior in traditional commodities markets is a fairly well-studied subject.
The earliest relevant literature dates back to the late 1980's \cite{bikhchandani}, and since then a number of experiments and theoretical models have observed and characterized the behavior.
These results established that herding does indeed exist and offer reasonable suggestions motivating and modeling the phenomena.
Bikchandania \& Sharma identify two main reasons why rational human investors choose to herd: for reputation and compensation purposes~\cite{bikhchandani}.
Reputation-based herding is motivated by a trader's lack of trust in his or her private signal. 
If the manager directing the trader to buy or sell is considered unreliable, the individual on the trading floor may follow his competitors for lack of better alternatives.
Compensation-based herding is motivated by a fear of doing poorly on the trade floor. 
Bikchandani and Sharma observe that when a trader's compensation is tied to his or her performance, they will buy with the crowd because this is seen as the ``safe'' move. 
Instead of using private information and making a risky, though potentially profitable investment, they hedge bets with the market.
Additionally, the authors separate the notion of ``spurious'' herding, where investors facing similar constraints happen upon the same decision~\cite[281]{bikhchandani}.
At any rate, there have been sophisticated advances in the study of herding in traditional markets.
We will recount a few of these in Section \ref{background}.

However, high-frequency trading (HFT) has revolutionized financial markets, and historical models may no longer apply.
At the very least, the time is ripe for new models which are specifically tailored to predict herding in a high-frequency setting.
While there is some literature on herding in HFT markets, it is relatively little and not quite as in-depth as past work on herding in traditional markets has been.
Our goal in this class has been to analyze the presence, significance, and implications of herd behavior in an HTF setting.

This paper is outlined as follows: Section \ref{background} gives more historical information and recounts the studies we found relevant to our work.
In Section \ref{observations}, we talk about the challenges and peculiarities we forsee in HFT markets.
That is, we give a high-level description of our intuitions regarding herding in high-frequency markets.
Section \ref{experiment} contains the main part of this work, introducing our data source and documenting our experiments.
Finally we conclude in Section \ref{conclusion} with a final overview of where we have come.
